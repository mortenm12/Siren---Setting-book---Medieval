\chapter{Campaign}
How you as a Game Master should consider making a campaign in this setting.

The creatures in the monster chapter can be used if you want to, or you can make your own.
I have drown inspiration from medieval mythical creatures, you can choose to extend this to more magical creatures, or you can only choose to choose real medieval creatures.

Whom does rule the area, kingdom, city or town the campaign is taking place in?
Is there a lot of lawless running around?
What about trading inside or outside of the kingdom, and what are traded?


\section{Treasure}
Treasure can be everything from better armour, weapon, or magical items as you prefer, or just gold and valuable stuff.
Even information can be enough treasure for some.

The treasure should be compared with what the characters did to get the treasure, or how experience the character is.
A powerful knight, do not need a new armour, especially is the new armour is weaker than the one he already have.
But a potion that can make him endured in battle, will maybe a better treasure for him.

\section{Challenges}
A usual starter character with 8 XP as its staring point, is not very skillful nor powerful.
The character is only a little bit more skillful or powerful than the average human in the world.
The character should instead be interested to go on an adventure, or to seek new places to discover.


A character at 8 XP usually have 5 and 6 in most of its traits, and maybe a single trait at 3 - 4.
This means that a group of 3-5 characters with 8 XP each, can not take upon the task to kill a Dragon, or another big and dangerous creature.
But a single knight, or a group of robbers would be more suited in challenges for the players.


If the players should be able to take upon the task to kill the dragon, then they need to be around 50 XP.

For the challenge comparison, you as a GM should hold count on how many XP you have given the players over time.
This will make it more manageable over time to make sure that the players is not in too deep danger, if you not wish to.

